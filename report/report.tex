\documentclass[10pt, oneside]{report}

\usepackage[utf8]{inputenc}
\usepackage{amsmath}
\usepackage{amssymb}
\usepackage{mathrsfs}
\usepackage{enumerate}

\begin{document}
\subsubsection{Question 1.} For $k\in\mathcal{K}, n\in\mathcal{N}, m=1,\dots,M$, we define the binary variables $x_{k,m,n}\in\{0,1\}$ such that $x_{k,m,n}=1$ if and only if $p_{k,m,n}\le p_{k,n}<p_{k,m+1,n}$, with $p_{k,M+1,n}$ interpreted as $+\infty$. Thus, the constraint by the total power budget is
$$
\sum_{\substack{k\in\mathcal{K}\\n\in\mathcal{N}\\m=1,\dots,M}}p_{k,m,n}x_{k,m,n}\le p
$$
the constraint that each channel serves one and only one user is
$$
\sum_{\substack{k\in\mathcal{K}\\m=1,\dots,M}}x_{k,m,n}=1, \, \forall n\in\mathcal{N}
$$
and the target function is
$$
U:=\sum_{\substack{k\in\mathcal{K}\\n\in\mathcal{N}\\m=1,\dots,M}}r_{k,m,n}x_{k,m,n}
$$

In all, we have the ILP below
\begin{eqnarray}
x_{k,m,n}&\in&\{0,1\} \\
\sum_{k,m,n}p_{k,m,n}x_{k,m,n}&\le& p \\
\sum_{k,m}x_{k,m,n}&=&1, \, \forall n\in\mathcal{N}
\end{eqnarray}
with target function
$$
U:=\sum_{k,m,n}r_{k,m,n}x_{k,m,n}
$$
the corresponding LP is obtained by replacing (1) with $x_{k,m,n}\in[0,1]$

\subsubsection{Question 2.} Proof of \textbf{Lemma 1}

Suppose $p_{k,m,n}\le p_{k',m',n}$ and $r_{k,m,n}\ge r_{k',m',n}$. Given an optimal solution to the ILP, if $x_{k',m',n}\ne0$, then $x_{k',m',n}=1$ by (1). If we replace $x_{k,m,n}$ by 1 and $x_{k',m',n}$ by 0, (1) and (3) are clearly satisfied, (2) is also satisfied since $p_{k,m,n}\le p_{k',m',n}$, and $U$ will not decrease since $r_{k,m,n}\ge r_{k',m',n}$, so we get an optimal where $x_{k',m',n}=0$

\hspace{1em}
\subsubsection{Question 3.}
\noindent\textsc{remove-IP-dominated}(n)
\begin{enumerate}[1\ ]
\setlength{\topsep}{0.05ex}
\setlength{\itemsep}{0.05ex}
\item sort the pairs $(p_{k,m,n}, r_{k,m,n})$ in increasing order of $p_{k,m,n}$ into an array A. If several pairs have the same $p$, leave only the one with the greatest $r$
\item cm = A[0].r
\item \textbf{for} i = 0 to A.length-1
\item \qquad \textbf{if} A[i].r $\ge$ cm
\item \qquad \qquad cm = A[i].r
\item \qquad \textbf{else} remove A[i].p and A[i].r from the original data
\end{enumerate}


sorting $p_{k,m,m}$ takes time $O(KM\log(KM))$, the loop from line 3 takes time $O(KM)$. We will run \textsc{remove-IP-dominated} for each $n\in\mathcal{N}$, so in all it takes time $O(NKM\log(KM))$ to remove IP-dominated terms.

\noindent\textsc{N.B.} Accounting for that $p_{k,m,n}$ is increasing according to $m$ for $k,n$ fixed, we may sort quicker in line 1 and achieve a complexity of $O(NKM\log(K)$.

\subsubsection{Question 4.}
Proof of \textbf{Lemma 2}

Suppose $p_{k,m,n}\le p_{k',m',n}\le p_{k'',m'',n}$ and
\begin{equation}
\frac{r_{k'',m'',n}-r_{k',m',n}}{p_{k'',m'',n}-p_{k',m',n}}\ge\frac{r_{k',m',n}-r_{k,m,n}}{p_{k',m',n}-p_{k,m,n}}
\end{equation}
note that we have
\begin{equation}
p_{k',m',n}=p_{k,m,n}\frac{p_{k'',m'',n}-p_{k',m',n}}{p_{k'',m'',n}-p_{k,m,n}}+p_{k'',m'',n}\frac{p_{k',m',n}-p_{k,m,n}}{p_{k'',m'',n}-p_{k,m,n}}
\end{equation}
and from (4) we can deduce that
\begin{equation}
r_{k',m',n}\le r_{k,m,n}\frac{p_{k'',m'',n}-p_{k',m',n}}{p_{k'',m'',n}-p_{k,m,n}}+r_{k'',m'',n}\frac{p_{k',m',n}-p_{k,m,n}}{p_{k'',m'',n}-p_{k,m,n}}
\end{equation}
Given an optimal solution to the LP, we can construct another solution with
\begin{align*}
x'_{k,m,n}&=x_{k,m,n}+x_{k',m',n}\frac{p_{k'',m'',n}-p_{k',m',n}}{p_{k'',m'',n}-p_{k,m,n}}\\
x'_{k',m',n}&=0\\
x'_{k'',m'',n}&=x_{k'',m'',n}+x_{k',m',n}\frac{p_{k',m',n}-p_{k,m,n}}{p_{k'',m'',n}-p_{k,m,n}}
\end{align*}
(3) is satisfied since $x'_{k,m,n}+x'_{k',m',n}+x'_{k'',m'',n}=x_{k,m,n}+x_{k',m',n}+x_{k'',m'',n}$, and so is (2) since $x'_{k,m,n},\,x'_{k',m',n},\,x'_{k'',m'',n}\ge0$ and none of them can surpass 1 or else one of the $x'{\cdot,\cdot,n}$ would be negative.

(2) is satisfied owing to (5), and $U'\ge U$ owing to (6). Thus we get an optimal solution where $x_{k',m',n}=0$

In the pseudo-code below, we consider the input as points in the plane with coordinates $(p_{k,m,n},r_{k,m,n})$. For simplicity, for a stack S, we note S[0] the top element and S[1] the second top one; for tow points $A,B$ in the plane, we note $L(A,B)$ the slope of the line formed between them; we say $B$ is \emph{dominated} by $A$ and $C$ if and only if $A.p\le B.p\le C.p$ and $L(A,B)\le L(B,C)$, which is just another interpretation of (4)

\hspace{1em}

\noindent\textsc{remove-LP-dominated}(n)
\begin{enumerate}[1\ ]
\setlength{\topsep}{0.05ex}
\setlength{\itemsep}{0.05ex}
\item sort the pairs $(p_{k,m,n}, r_{k,m,n})$ in increasing order of $p_{k,m,n}$ into an array A. If several pairs have the same $p$, leave only the one with the greatest $r$
\item let S be a stack
\item \textsc{Push}(A[0], S)
\item \textsc{Push}(A[1], S)
\item \textbf{for} i = 2 to A.length-1
\item \qquad \textbf{while} $L$(A[i],S[0]) $\ge$ $L$(S[0],S[1])
\item \qquad \qquad\textsc{Pop}(S)
\item \qquad \textsc{Push}(A[i], S)
\item \textbf{return} S
\end{enumerate}

\noindent\textbf{Proof of correctness} We use the loop invariant that after each iteration of line 5, S contains exactly all the points that are not dominated by points from A[0] to A[i], the line segments form a convex curve.

The invariant trivially holds before line 5. Suppose it holds before the ith iteration. During the ith iteration, the points removed by line 7 are clearly dominated by A[i] and S[1]. After the \textbf{while} loop terminates, we can be sure that all points in A between S[0] and A[i] are dominated by these two points and thus cannot be added to S, nor can the points between A[0] and S[0] by the loop invariant.

Note that the ponits in S form a convex curve, so that by $L$(A[i],S[0]) $<$ $L$(S[0],S[1]) we have that after line 8 the points in S still form a convex curve and so no points in S are dominated by points from A[0] to A[i], and by the arguments above, these are exactly all the points that have this property.

\hspace{1em}

\noindent\textbf{Time complexity} Line 1 takes time $O(KM\log(KM))$. Regarding the \textbf{for} loop from line 5 to 8, note that each point can only be pushed or popped only once, so in all the \textbf{for} loop takes time $O(KM)$. We run the algorithm for each $n\in\mathcal{N}$, so altogether it takes time $O(NKM\log(KM))$ to remove all LP-dominated terms.


\end{document}
